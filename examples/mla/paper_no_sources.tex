\documentclass[12pt,letterpaper]{article}

\usepackage{mla13}

\firstname{Sam}
\lastname{Craig}
\professor{Mrs. Mills}
\class{English 9 Honors}
\title{``The Most Dangerous Game'' Written Response}

\begin{document}
  
  General Zaroff, in Richard Conell's short story ``The Most Dangerous Game,'' has an interesting philosophy in life that Rainsford cannot agree with.
  General Zaroff believes that killing fellow humans for a challenging hunt is completely fine and moral because life is to be lived by the strong.
  This is a philosophy that many famous people have adopted---the idea of social Darwinism---just not in the way General Zaroff chooses to.
  For that reason, General Zaroff is following his philosophy as he hunts humans, just in the most radical way possible.
  This response will explain how General Zaroff is following his philosophy as he hunts humans, and is not completely insane.
  
  The idea of social Darwinism is common among the most powerful members of the world.
  For example, most of the powerful people in industrial America, such as Andrew Carnegie, John D. Rockefeller, or Cornelius Vanderbilt, would do something similar to General Zaroff, but using companies instead of humans.
  They would buy out competing companies in order to keep their company at the top.
  Normally they would only do this once the opposing company was big enough to challenge them.
  This relates exactly to what General Zaroff would do.  General Zaroff would kill humans as game, possibly to make himself feel at the top, and he would do so more happily if the human he was hunting would challenge him.
  The only reason he started hunting humans in the first place was, as he said about hunting, ``It had become too easy.
  I always got my quarry.
  Always.
  There is no greater bore than perfection.'' (224)
  He was always looking for a challenge before he hunted the humans, just like how the powerful people of industrial America only bought out the companies that would provide them a challenge.

  This, in turn, leads to the conclusion that what General Zaroff did is just a very radical form of social Darwinism.
  Where many humans that believed in social Darwinism would let the weak die off on their own, General Zaroff took it into his own hands.
  His actions can be related to many powerful people.
  His actions were just much more radical than the actions of other powerful people of history.

\end{document}
